\documentclass{article}

\usepackage{setspace}
\usepackage{listings}
\usepackage{verbatim}

\usepackage{amsmath}
\DeclareMathOperator{\arcsec}{arcsec}
\DeclareMathOperator{\arccot}{arccot}
\DeclareMathOperator{\arccsc}{arccsc}
\DeclareMathOperator{\arcsech}{arcsech}
\DeclareMathOperator{\arccoth}{arccoth}
\DeclareMathOperator{\arccsch}{arccsch}
\DeclareMathOperator{\arcsinh}{arcsech}
\DeclareMathOperator{\arccosh}{arccoth}
\DeclareMathOperator{\arctanh}{arccsch}
\DeclareMathOperator{\sech}{sech}
\DeclareMathOperator{\csch}{csch}


\begin{document}
\section{Usage}
\subsection{Input}

Upon calling \textbf{makepoints}, print a number 0 or 1 to \textbf{cin}. 0 is for generating a slope field, 1 is for generating a curve.\\
To generate a slope field, enter the following arguments in this order:

\begin{verbatim}
string rpn #a string containing a postfix
expression, according to the definition below
#the next four variables define a section of
the Cartesian Plane to hold the slope field
double xmin
double xmax
double ymin
double ymax
int xs #the number of columns in the slope field
int ys #the number of rows in the slope field
\end{verbatim}

And to generate a curve:
\begin{verbatim}
string rpn
double xmin
double xmax
double ymin
double ymax
double initx #the x-coordinate of the initial condition
double inity #the y-coordinate of the initial condition
double samples #the number of line segments making up the curve
double len #the length of the desired curve
\end{verbatim}
\subsection{Output}
All input is printed to \textbf{cout}.

If a slope field was generated, then the output will be a list of numbers, corresponding to slopes. They are ordered as follows: First is the slope corresponding to the bottom-leftmost point in the slope field. From there they increase in $y$ but not in $x$, \textbf{ys} times. Then it returns to the bottom and increments $x$ by 1, and so on, going up each column in turn. Each slope is separated by a new line. Keep in mind that $\pm$\textbf{nan} and $\pm$\textbf{inf} are possible outputs and must be dealt with.

If a curve was generated, then the output will be a list of pairs of numbers. Within pairs, numbers are separated by spaces, and pairs are separated by new lines. The first element of a pair corresponds to its relative $x$-position in the specified range; -1 is the furthest left $x$ value, and 1 is the furthest right. Similarly, the second corresponds to its relative $y$-position, again from -1 to 1. \textit{Keep in mind that $|x|$ or $|y|$ can be greater than $1$; this indicates that it is outside of the specified range.}The pairs are all sorted from smallest to greatest value of $x$. The curve is generated by ``connecting the dots'' in order that they are outputted. Termination of the output is marked by the pair \textbf{inf inf}.
\section{Postfix String}
\subsection{Functions}
{\doublespacing
\subsubsection{Nice Functions}
\begin{tabular}{lll}
$x$ $y$ $+$ = $x + y$ & $x$ $y$ $-$ $=$ $x - y$ & $x$ $y$ $*$ $=$ $x \times y$\\
$x$ $y$ $/$ $=$ $\frac{x}{y}$&$x$ $y$ \^{} $=$ $x^y$&$x$ S $=$ $\sqrt{x}$\\
$x$ l $=$ $\ln(x)$& $x$ a $=$ $|x|$&\\
\end{tabular}
\subsubsection{Trig Functions}
\begin{tabular}{lll}
$x$ s $=$ $\sin(x)$&$x$ c $=$ $\cos(x)$& $x$ t $=$ $\tan(x)$\\
$x$ u $=$ $\arcsin(x)$&$x$ v $=$ $\arccos(x)$& $x$ w $=$ $\arctan(x)$\\
$x$ p $=$ $\sec(x)$&$x$ q $=$ $\csc(x)$& $x$ r $=$ $\cot(x)$\\
$x$ d $=$ $\arcsec(x)$&$x$ J $=$ $\arccsc(x)$& $x$ f $=$ $\arccot(x)$\\
$x$ h $=$ $\sinh(x)$&$x$ i $=$ $\cosh(x)$& $x$ j $=$ $\tanh(x)$\\
$x$ m $=$ $\arcsinh(x)$&$x$ n $=$ $\arccosh(x)$& $x$ o $=$ $\arctanh(x)$\\
$x$ g $=$ $\sech(x)$&$x$ k $=$ $\csch(x)$& $x$ z $=$ $\coth(x)$\\
$x$ A $=$ $\arcsech(x)$&$x$ B $=$ $\arccsch(x)$& $x$ C $=$ $\arccoth(x)$\\
\end{tabular}
}\subsection{Requirements}{
\begin{itemize}
\item{The RPN expression must have every argument separated by one space.}
\item{The RPN expression must have a trailing space.}
\end{itemize}
}
\end{document}
